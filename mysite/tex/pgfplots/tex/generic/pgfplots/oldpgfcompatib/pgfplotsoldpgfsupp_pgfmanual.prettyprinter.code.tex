%%%%%%%%%%%%%%%%%%%%%%%%%%%%%%%%%%%%%%%%%%%%%%%%%
%%% This file is a copy of some part of PGF/Tikz.
%%% It has been copied here to provide :
%%%  - compatibility with older PGF versions
%%%  - availability of PGF contributions by Christian Feuersaenger
%%%    which are necessary or helpful for pgfplots.
%%%
%%% For reasons of simplicity, I have copied the whole file, including own contributions AND
%%% PGF parts. The copyrights are as they appear in PGF.
%%%
%%% Note that pgfplots has compatible licenses.
%%%
%%% This copy has been modified in the following ways:
%%%  - nested \input commands have been updated
%%%
%
% Support for the contents of this file will NOT be done by the PGF/TikZ team.
% Please contact the author and/or maintainer of pgfplots (Christian Feuersaenger) if you need assistance in conjunction
% with the deployment of this patch or partial content of PGF. Note that the author and/or maintainer of pgfplots has no obligation to fix anything:
% This file comes without any warranty as the rest of pgfplots; there is no obligation for help.
%%%%%%%%%%%%%%%%%%%%%%%%%%%%%%%%%%%%%%%%%%%%%%%%%%
%%% Date of this copy: Sa 7. Dez 20:58:23 CET 2013 %%%



%--------------------------------------------
%
% Package pgfplots
%
% Provides a user-friendly interface to create function plots (normal
% plots, semi-logplots and double-logplots).
%
% It is based on Till Tantau's PGF package.
%
% Copyright 2007/2008/2009 by Christian Feuersänger.
%
% This program is free software: you can redistribute it and/or modify
% it under the terms of the GNU General Public License as published by
% the Free Software Foundation, either version 3 of the License, or
% (at your option) any later version.
%
% This program is distributed in the hope that it will be useful,
% but WITHOUT ANY WARRANTY; without even the implied warranty of
% MERCHANTABILITY or FITNESS FOR A PARTICULAR PURPOSE.  See the
% GNU General Public License for more details.
%
% You should have received a copy of the GNU General Public License
% along with this program.  If not, see <http://www.gnu.org/licenses/>.
%
%--------------------------------------------

% this is some kind of "plug-in" for pgfmanual-en-macros.tex which
% provides pretty printing of the code lines in 'codeexample'.
%
% It requires to be invoked with \pgfmanualprettyprintcode{#1} at the
% right place.
%
% ATTENTION:
% this is NOT a sophisticated syntax highlighter like lstlistings!
% It has rather strict assumptions about how to use it (the input must
% have been read verbatim, for example).

% Special code for syntax highlighting:
%
%
% USER INTERFACE
\pgfkeys{%
	% this is the public hook into
	% \begin{codeexample}...\end{codeexample} which triggers pretty
	% printing:
	/codeexample/typeset listing/.code=	{\pgfmanualprettyprintcode{#1}},
	%
	% this key will be invoked whenever a key in key-value context has been identified.
	%
	% It won't be invoked for handled keys like |my style/.style=....|, see below.
	%
	% #1 will be the keys name.
	/codeexample/prettyprint/key name/.code={#1},
	%
	% A variant which will be used instead of |key name| if the key
	% has a key handler. For example, when the pretty printer finds
	% |my style/.style|, it will call
	% |key name with handler={my style}{.style}.
	/codeexample/prettyprint/key name with handler/.code 2 args={#1/#2},
	%
	% Will be invoked whenever the value of a key has been found.
	% It will be invoked AFTER 'key name' and not at all if there was
	% no value.
	%
	% #1: the key's name
	% #2: the key's value.
	%
	% The default implementation checks if there is a special handler
	% for the key name '#1', in
	%    /codeexample/prettyprint/key value/#1.
	% For example:
	%    /codeexample/prettyprint/key value/my key/.code 2 args={name=#1, value =#2}
	% If such a key exists, it will be invoked with {#1}{#2} as
	% arguments. Otherwise, the generic method
	% 	/codeexample/prettyprint/key value with style detection={#1}{#2}
	% will be invoked.
	/codeexample/prettyprint/key value/.code 2 args={%
		\pgfkeysifdefined{/codeexample/prettyprint/key value/#1}{%
			\pgfkeysalso{/codeexample/prettyprint/key value/#1={#1}{#2}}%
		}{%
			\pgfkeysalso{/codeexample/prettyprint/key value with style detection={#1}{#2}}%
		}%
	},
	% A helper macro for the default 'key value' implementation.
	/codeexample/prettyprint/key value with style detection/.code 2 args={%
		\pgfutil@in@{style}{#1}%
		\ifpgfutil@in@
			\pgfmanualprettyprintpgfkeys{#2}%
		\else
			\pgfkeysalso{/codeexample/prettyprint/key value display only={#2}}%
		\fi
	},%
	%
	% Will be invoked by the default 'key value' implementation to
	% typeset the value as such.
	/codeexample/prettyprint/key value display only/.code={#1},
	%
	% Used to typeset a single word. A word is something which has not
	% been identified in any other context; a maximal sequence of
	% non-white-space tokens.
	/codeexample/prettyprint/word/.code={#1},
	%
	% A two-argument code key which invokes '#1' if spaces shall
	% delimit words and '#2' if not.
	% This may be useless. Handle with care!
	/codeexample/prettyprint/if space is word delim/.code 2 args={%
		#1%
	},%
	%
	% A three-argument code key which should invoke '#2' if the single
	% token '#1' is a word delimiter and '#3' if that is not the case.
	% Note that '#1' doesn't have the usual catcodes (see the
	% \pgfmanual@pretty@** macros)!
	% Furthermore, white spaces are treated separately.
	/codeexample/prettyprint/if is word delim/.code args={#1#2#3}{%
		\edef\pgfmanual@check{,;().;\pgfmanual@pretty@lbrace\pgfmanual@pretty@rbrace}%
		\def\pgfmanual@check@{\pgfutil@in@{#1}}%
		\expandafter\pgfmanual@check@\expandafter{\pgfmanual@check}%
		\ifpgfutil@in@ #2\else #3\fi
	},%
	%
	% Used to typeset a single control sequence.
	% #1 is the control sequence's name as a sequence of catcode 12
	% tokens.
	/codeexample/prettyprint/cs/.code={#1},
	%
	% Used to typeset a single control sequence together with one or
	% more arguments.
	%
	% #1 is the control sequence's name (including the backslash as
	% literal string) and
	% #2,#3,...,#9 are the arguments. The number of arguments depends
	% on the 'cs arguments/<CS NAME>' key; it contains the integer
	% number.
	%
	% For example, if
	% |cs arguments/pgfkeys/.initial=1|,
	% the pretty printer invokes
	% |cs with args={\pgfkeys}{<argument>}.
	%
	% If
	% |cs arguments/mycommand/.initial=2|,
	% the pretty printer invokes
	% |cs with args={\pgfkeys}{<argument1>}{<argument2>}.
	% In this case, 'cs with args' needs to be defined in a way which
	% allows to deal with 3 arguments.
	%
	% Please note that the value do not contain braces! The braces
	% need to be inserted manually.
	%
	% Special cases:
	% 1. If end-of-input is encountered BEFORE the expected number of
	% arguments has been found, the following happens:
	% - if not even one (partial) argument has been found, the
	%   /codeexample/prettyprint/cs/.@cmd key will be used instead.
	% - the \ifpgfmanualprettyprinterarghasunmatchedbraces boolean
	%   indicates if the trailing brace is missing.
	/codeexample/prettyprint/cs with args/.code 2 args={#1\{#2\pgfmanualclosebrace},
	%
	% For every identified control sequence, the key
	%    /codeexample/prettyprint/cs arguments/<CSNAME>
	% will be checked. If it exists, it contains an integer with the
	% number of arguments it takes.
	% The <CSNAME> must not contain the '\'.
	/codeexample/prettyprint/cs arguments/pgfkeys/.initial=1,
	%
	% For every identified control sequence, the key
	%	/codeexample/prettyprint/cs/<CSNAME>/.@cmd
	% will be checked. If it exists, it will be used instead of
	% 'cs with args' (with the same special case restrictions for
	% early end-of-input).
	%
	% The <CSNAME> must not contain the '\'.
	%
	% This allows special treatment for special
	% commands like \pgfkeys:
	/codeexample/prettyprint/cs/pgfkeys/.code 2 args={#1\{\pgfmanualprettyprintpgfkeys{#2}\pgfmanualclosebrace},
	%
	% recognise environments:
	/codeexample/prettyprint/cs arguments/begin/.initial=1,
	/codeexample/prettyprint/cs/begin/.code 2 args={#1\{#2\pgfmanualclosebrace},
	/codeexample/prettyprint/cs arguments/end/.initial=1,
	/codeexample/prettyprint/cs/end/.code 2 args={#1\{#2\pgfmanualclosebrace},
	%
	% a point coordinate (#1)
	% Note that '#1' will contain the braces (if there was one),
	/codeexample/prettyprint/point/.code={#1},%
	%
	% a point coordinate with explicit coordinate system: (#1:#2)
	% Note that '#2' will contain the terminating brace (if there was one)
	% #1: the coordinate system name (*without* the leading brace).
	% The leading brace must be re-inserted by this style.
	/codeexample/prettyprint/point with cs/.code 2 args={(#1:#2},%
	% the same with specialised code:
	%/codeexample/prettyprint/point with cs/<CS NAME>/.code 2 args={(#1:#2},%
	%
	%
	% A predefined style which colors every identified token. It's
	% purpose is only debugging.
	/codeexample/prettyprint/colored/.style={%
		/codeexample/prettyprint/key name/.code={\textcolor{red!75!black}{##1}},
		/codeexample/prettyprint/key name with handler/.code 2 args={\textcolor{red!75!black}{##1}/\textcolor{red!90!black}{##2}},
		/codeexample/prettyprint/key value display only/.code={\textcolor{yellow}{##1}},
		/codeexample/prettyprint/cs/.code={\textcolor{green}{##1}},
		/codeexample/prettyprint/cs with args/.code 2 args={\textcolor{green}{##1}\{\textcolor{orange}{##2}\pgfmanualclosebrace},
		/codeexample/prettyprint/cs arguments/pgfkeys/.initial=1,
		/codeexample/prettyprint/cs/pgfkeys/.code 2 args={\textcolor{green}{##1}\{\pgfmanualprettyprintpgfkeys{##2}\pgfmanualclosebrace},
		/codeexample/prettyprint/cs arguments/begin/.initial=1,
		/codeexample/prettyprint/cs/begin/.code 2 args={\textcolor{green}{##1}\{\textcolor{orange}{##2}\pgfmanualclosebrace},
		/codeexample/prettyprint/cs arguments/end/.initial=1,
		/codeexample/prettyprint/cs/end/.code 2 args={\textcolor{green}{##1}\{\textcolor{orange}{##2}\pgfmanualclosebrace},
		/codeexample/prettyprint/word/.code={\textcolor{brown}{##1}},
		/codeexample/prettyprint/point/.code={\textcolor{red}{##1}},%
		/codeexample/prettyprint/point with cs/.code 2 args={(\textcolor{green}{##1}:\textcolor{red}{##2}},%
	},%
	% A further debuggin helper.
	/codeexample/prettyprint/colored and verbose/.style={%
		/codeexample/prettyprint/colored,
		/codeexample/prettyprint/key name/.code={\message{[key name=##1]}\textcolor{red!75!black}{##1}},
		/codeexample/prettyprint/key name with handler/.code 2 args={\message{[key name with handler=##1/##2]}\textcolor{red!75!black}{##1}/\textcolor{red!90!black}{##2}},
		/codeexample/prettyprint/key value display only/.code={\def\temp{##1}\message{[key value=\meaning\temp]}\textcolor{yellow}{##1}},
		/codeexample/prettyprint/cs/.code={\message{[cs=##1]}\textcolor{green}{##1}},
		/codeexample/prettyprint/cs with args/.code 2 args={\def\temp{##2}\message{[cs with args={##1}{\meaning\temp}]}\textcolor{green}{##1}\{\textcolor{orange}{##2}\pgfmanualclosebrace},
		/codeexample/prettyprint/word/.code={\message{[word=##1]}\textcolor{brown}{##1}},
		/codeexample/prettyprint/point/.code={\message{[point=##1]}\textcolor{red}{##1}},%
		/codeexample/prettyprint/point with cs/.code 2 args={\message{[point with cs={##1}{##2}]}(\textcolor{green}{##1}:\textcolor{red}{##2}},%
	},%
	%/codeexample/prettyprint/colored,
	%/codeexample/prettyprint/colored and verbose,
}%

\newif\ifpgfmanualprettyenabled
\pgfmanualprettyenabledtrue

% User interface command to handle unmatched braces:
%
% It expands to '\}'  unless the preceding argument had unmatched braces.
\def\pgfmanualclosebrace{%
	\ifpgfmanualprettyprinterarghasunmatchedbraces
	\else
		\}%
	\fi
}%


% Typesets '#1', possibly using some sort of pretty printer.
%
% The argument '#1' is expected to be a long token list in which
% 1. all characters have catcode 11 or 12 (normal letters),
% 2. white spaces, tabs and newlines have catcode 13 (are active).
%
% @see \pgfmanualprettyprintpgfkeys
\long\def\pgfmanualprettyprintcode#1{%
%\def\temp{#1}\message{STARTING PRETTY PRINTING for \meaning\temp}%
	\ifpgfmanualprettyenabled
		\begingroup
		\pgfmanualprettyprintstyles
		\pgfmanual@pretty@mainloop#1\pgfmanual@EOI
		\endgroup
	\else
		#1%
	\fi
}%
% DISABLE:
%\long\def\pgfmanualprettyprintcode#1{#1}%

\def\pgfmanualprettyprintstyles{%
%\tracingmacros=2 \tracingcommands=2 \message{PRETTYPRINT INIT}%
	\def\{{\pgfmanual@pretty@lbrace}%
	\def\}{\pgfmanual@pretty@rbrace}%
	\pgfmanual@pretty@installcommenthandler
	\let\pgfmanualprettyprintstyles=\relax
}%


\newif\ifpgfmanualprettycommentactive

% we can't use the \@typeset@till@return method for comments because
% the pretty printer needs full control over the token list.
%
% We try a replacement here.
%
% FIXME
% THIS INTRODUCES A NEW GROUP! Does that hurt the pretty printer??
%
\def\pgfmanual@pretty@installcommenthandler{%
	\expandafter\def\pgfmanual@pretty@activepercent%
		%\tt
		\% %
	}%
}%
\def\pgfmanual@pretty@recoveraftercomment{%
	\endgroup
	\pgfmanual@pretty@activenl
}%

\def\pgfmanual@EOI{\pgfmanual@EOI}%

{
	\catcode`\[=0
	\catcode`\\=12
	[gdef[pgfmanual@pretty@backslash{\}%
}
\begingroup
	\catcode`\:=12
	\catcode`\(=12
	\catcode`\)=12
	\gdef\pgfmanual@pretty@colon{:}%
	\gdef\pgfmanual@pretty@lroundbrace{(}%
	\gdef\pgfmanual@pretty@rroundbrace{)}%
	\catcode`\[=1
	\catcode`\]=2
	\catcode`\{=12
	\catcode`\}=12
	\gdef\pgfmanual@pretty@lbrace[{]%
	\gdef\pgfmanual@pretty@rbrace[}]%
	\catcode`\{=1
	\catcode`\}=2
	\catcode`\[=12
	\catcode`\]=12
	\catcode`\,=12
	\catcode`\ =10\relax\gdef\pgfmanual@pretty@space{ }%
	\gdef\pgfmanual@pretty@lbracket{[}%
	\gdef\pgfmanual@pretty@rbracket{]}%
	\gdef\pgfmanual@pretty@comma{,}%
	\catcode`\==12
	\gdef\pgfmanual@pretty@eq{=}%
\endgroup
\begingroup
	\catcode`\%=12
	\gdef\pgfmanual@pretty@percent{%}
	\catcode`\%=13
	\gdef\pgfmanual@pretty@activepercent{%}
	\catcode`\^^M=13
	\gdef\pgfmanual@pretty@activenl{^^M}\endgroup
\begingroup
\catcode`\^^I=13\relax\gdef\pgfmanual@pretty@activetab{^^I}%
\catcode`\ =13\relax\gdef\pgfmanual@pretty@activespace{ }\endgroup

% loops through all tokens.
% #1 is a single token.
\def\pgfmanual@pretty@mainloop{%
	\def\pgfmanual@pretty@mainloop@currentword{}%
	\pgfmanual@pretty@mainloop@
}%
\long\def\pgfmanual@pretty@mainloop@#1{%
	\def\pgfmanual@loc@TMPa{#1}%
	\let\pgfmanual@pretty@mainloop@NEXT=\pgfmanual@pretty@mainloop@
	\ifx\pgfmanual@loc@TMPa\pgfmanual@EOI
		% stop iteration.
		\pgfmanual@pretty@mainloop@finishword
		\let\pgfmanual@pretty@mainloop@NEXT=\relax
	\else
		\ifpgfmanualprettycommentactive
			#1%
		\else
			\ifx\pgfmanual@loc@TMPa\pgfmanual@pretty@lbracket%
				% we found the start of OPTIONS '[ ... ]'.
				\pgfmanual@pretty@mainloop@finishword
				% Attempt syntax highlighting for pgfkeys:
				\let\pgfmanual@pretty@mainloop@NEXT=\pgfmanual@pretty@pgfkeys
			\else
				\ifx\pgfmanual@loc@TMPa\pgfmanual@pretty@backslash
					% we found the start of a control sequence '\command'
					\pgfmanual@pretty@mainloop@finishword
					%
					% collect the control sequence name into
					% \pgfmanual@loc@TMPa, but without the backslash:
					\let\pgfmanual@loc@TMPa=\pgfutil@empty
					\let\pgfmanual@pretty@mainloop@NEXT=\pgfmanual@pretty@collectcs@loop
				\else
					\ifx\pgfmanual@loc@TMPa\pgfmanual@pretty@lroundbrace%
						\pgfmanual@pretty@mainloop@finishword
						\let\pgfmanual@pretty@mainloop@NEXT=\pgfmanual@pretty@collectpoint
					\else
						\def\pgfmanual@isspace{0}%
						\ifnum13=\catcode`#1
							% we found a white space (space, TAB or NL) or comment
							\def\pgfmanual@isspace{1}%
						\else
							\ifnum10=\catcode`#1
								\def\pgfmanual@isspace{1}%
							\fi
						\fi
						\if\pgfmanual@isspace1%
							\pgfkeysvalueof{/codeexample/prettyprint/if space is word delim/.@cmd}{%
								\pgfmanual@pretty@mainloop@finishword
								#1% ok, show it.
							}{%
								% collect the word.
								\expandafter\def\expandafter\pgfmanual@pretty@mainloop@currentword\expandafter{\pgfmanual@pretty@mainloop@currentword #1}%
							}%
						\else
							\pgfkeysvalueof{/codeexample/prettyprint/if is word delim/.@cmd}{#1}{%
								\pgfmanual@pretty@mainloop@finishword
								#1% ok, show the delimiter.
							}{%
								% collect the word.
								\expandafter\def\expandafter\pgfmanual@pretty@mainloop@currentword\expandafter{\pgfmanual@pretty@mainloop@currentword #1}%
							}\pgfeov%
						\fi
					\fi
				\fi
			\fi
		\fi
	\fi
	\pgfmanual@pretty@mainloop@NEXT
}%

\def\pgfmanual@pretty@mainloop@finishword{%
	\ifx\pgfmanual@pretty@mainloop@currentword\pgfutil@empty
	\else
		\def\pgfmanual@loc@TMPa{\pgfkeysvalueof{/codeexample/prettyprint/word/.@cmd}}%
		\expandafter\pgfmanual@loc@TMPa\pgfmanual@pretty@mainloop@currentword\pgfeov
		\let\pgfmanual@pretty@mainloop@currentword=\pgfutil@empty
	\fi
}%

\def\pgfmanual@pretty@collectpoint{%
	\let\pgfmanualprettyprinterhandlecollectedargs=\pgfmanual@pretty@collectargs@finish@collectpoint
	\expandafter\pgfmanualprettyprintercollectupto\pgfmanual@pretty@rroundbrace
		{\afterpgfmanual@pretty@collectpoint}%
}

{
\catcode`\:=12
\gdef\pgfmanual@pretty@collectargs@finish@collectpoint#1{%
	\expandafter\pgfutil@in@\pgfmanual@pretty@colon{#1}%
	\ifpgfutil@in@
		\def\pgfmanual@pretty@collectpoint@getcoordsystem##1:##2\relax{%
			\begingroup
				\pgfmanual@pretty@restorespaces
				% remove active spaces from ##1:
				\xdef\pgfmanual@pretty@glob@TMPa{##1}%
			\endgroup
			\pgfkeysifdefined{/codeexample/prettyprint/point with cs/\pgfmanual@pretty@glob@TMPa/.@cmd}{%
				\def\pgfmanual@loc@TMPb{\pgfkeysvalueof{/codeexample/prettyprint/point with cs/\pgfmanual@pretty@glob@TMPa/.@cmd}}%
			}{%
				\def\pgfmanual@loc@TMPb{\pgfkeysvalueof{/codeexample/prettyprint/point with cs/.@cmd}}%
			}%
			\expandafter\def\expandafter\pgfmanual@loc@TMPa\expandafter{\expandafter{\pgfmanual@pretty@glob@TMPa}{##2}}%
			\expandafter\pgfmanual@loc@TMPb\pgfmanual@loc@TMPa\pgfeov
		}%
		\ifpgfmanualprettyprinterfoundterminator
			\pgfmanual@pretty@collectpoint@getcoordsystem#1)\relax
		\else
			\pgfmanual@pretty@collectpoint@getcoordsystem#1\relax
		\fi
	\else
		\def\pgfmanual@loc@TMPb{\pgfkeysvalueof{/codeexample/prettyprint/point/.@cmd}}%
		\ifpgfmanualprettyprinterfoundterminator
			\pgfmanual@loc@TMPb(#1)\pgfeov
		\else
			\pgfmanual@loc@TMPb(#1\pgfeov
		\fi
	\fi
}%
}
\def\afterpgfmanual@pretty@collectpoint{\pgfmanual@pretty@mainloop}%

% loops through all tokens, assembling the cs name as it goes.
% #1 is a single token.
\def\pgfmanual@pretty@collectcs@loop#1{%
	\let\pgfmanual@pretty@collectcs@loop@NEXT=\pgfmanual@pretty@collectcs@loop
	\def\pgfmanual@loc@TMPb{#1}%
	\ifx\pgfmanual@loc@TMPb\pgfmanual@EOI
		\def\pgfmanual@pretty@collectcs@loop@NEXT{\pgfmanual@pretty@collectcs@loop@END #1}%
	\else
		% a csname may only use letters. And: only letters have a
		% positive lcccode!
		\ifnum\lccode`#1=0
			\ifx\pgfmanual@loc@TMPb @%
				% ok, we may also accept an `@':
				\edef\pgfmanual@loc@TMPa{\pgfmanual@loc@TMPa #1}%
			\else
				% This here is the first non-letter.
				\def\pgfmanual@pretty@collectcs@loop@NEXT{\pgfmanual@pretty@collectcs@loop@END #1}%
			\fi
		\else
			% continue iterating and assemble the csname...
			\edef\pgfmanual@loc@TMPa{\pgfmanual@loc@TMPa #1}%
		\fi
	\fi
	\pgfmanual@pretty@collectcs@loop@NEXT
}%
\def\afterpgfmanual@pretty@cs{\pgfmanual@pretty@mainloop}%

\def\pgfmanual@pretty@collectcs@loop@END{%
	\pgfkeysifdefined{/codeexample/prettyprint/cs arguments/\pgfmanual@loc@TMPa}{%
		% oh. The collected control sequence expects arguments. That
		% means a lot of work since there are no nestable braces in
		% the token list! All of them have catcode 12... we need to do
		% that manually.
		\let\pgfmanualprettyprinterhandlecollectedargs=\pgfmanual@pretty@collectargs@finish@cs
		\def\pgfmanual@pretty@collectcs@loop@END@next{%
			\pgfmanualprettyprintercollectargcount
				{\pgfkeysvalueof{/codeexample/prettyprint/cs arguments/\pgfmanual@loc@TMPa}}%
				{\afterpgfmanual@pretty@cs}%
		}%
	}{%
		% re-insert the backslash:
		\edef\pgfmanual@loc@TMPa{\pgfmanual@pretty@backslash \pgfmanual@loc@TMPa}%
		% ok, report the macro and continue with the main loop.
		\def\pgfmanual@loc@TMPb{\pgfkeysvalueof{/codeexample/prettyprint/cs/.@cmd}}%
		\expandafter\pgfmanual@loc@TMPb\pgfmanual@loc@TMPa\pgfeov
		\let\pgfmanual@pretty@collectcs@loop@END@next=\afterpgfmanual@pretty@cs
	}%
	\pgfmanual@pretty@collectcs@loop@END@next
}%

\newif\ifpgfmanualprettyprinterarghasunmatchedbraces

% Collects arguments inside of a token list, dealing with nested
% catcode-12-braces.
%
% #1: is the NUMBER of arguments to collect.
% #2: is TeX code which shall be invoked after
% \pgfmanualprettyprinterhandlecollectedargs has been invoked (see
% below).
%
% PRECONDITION:
% 	there is a large token list following
% 	\pgfmanualprettyprintercollectargcount with balanced braces. The braces
% 	have \catcode 12.
% POSTCONDITION:
% 	1. the arguments will be collected as
% 		'<result> := {<first arg>}{<second arg>}',
% 	2.1 the macro \pgfmanualprettyprinternumcollectedargs will contain
% 	the *actual* number of *completely* collected arguments,
%	2.2 the if \ifpgfmanualprettyprinterarghasunmatchedbraces will be
%	set to \c true if the last found argument had an unmatched brace,
% 	3. then, \pgfmanualprettyprinterhandlecollectedargs{<result>} will be
% 	invoked. Just define \pgfmanualprettyprinterhandlecollectedargs
% 	properly.
% 	Afterwards, #2 will be invoked to recover from the argument
% 	collection.
%
%
% Example:
% 	\pgfmanualprettyprintercollectargcount{2}{\donnextstep}
% 		{_12 first argument }_12 {_12 second argument }_12 next tokens%
%
% 	will result in the expansion
% 		\pgfmanualprettyprinterhandlecollectedargs{{_1 first argument }_2 {_1 second argument }_2 }
%		\donnextstep
% 		next tokens
%
%
%
% ATTENTION:
% 	\pgfmanualprettyprinterhandlecollectedargs will be invoked with a SINGLE
% 	argument. The argument as such will contain extra braces, one pair
% 	of braces for each of the #1 arguments. These braces will have
% 	catcode 1 and 2, i.e. they can *really* be used in TeX. Thus, the
% 	finish routine will be invoked with
% 		\pgfmanualprettyprinterhandlecollectedargs
%	for one argument and with
% 		\pgfmanualprettyprinterhandlecollectedargs{{<first arg>}{<second arg>}}
%   for two and more arguments. Note the extra set of braces for one
%   arg.
%   It might happen that not all arguments have been found if
%   end-of-input occured before. The
%   \pgfmanualprettyprinterhandlecollectedargs routine has to check
%   this using the macro \pgfmanualprettyprinternumcollectedargs which
%   contains the actual number of collected args.
%
% @remark This macro checks for the end-of-input indicator, \pgfmanual@EOI. In case it encounters
% \pgfmanual@EOI, it
% 1. stops collecting, leaving the \pgfmanual@EOI as next token to be read,
% 2. sets \ifpgfmanualprettyprinterarghasunmatchedbraces to \iftrue if necessary,
% 3. assigns \pgfmanualprettyprinternumcollectedargs
% 4. invokes the finish routine.
% The \pgfmanual@EOI needs to be collected by following routines in
% this case. You can also use \pgfutil@ifnextchar\pgfmanual@EOI{}{} in
% following routines to check for \pgfmanual@EOI.
\def\pgfmanualprettyprintercollectargcount#1#2{%
	\pgfmanualprettyprinterarghasunmatchedbracesfalse
	\begingroup
	\edef\pgfmanual@loc@csargcount{#1}%
	\toksdef\t@afterpgfmanualprettyprinterhandlecollected=10
	\t@afterpgfmanualprettyprinterhandlecollected={#2}%
	\let\c@pgfmanual@pretty@openbracecount=\c@pgf@counta
	\let\c@pgfmanual@pretty@curargcount=\c@pgf@countb
	% I will track open braces and the number of completely
	% collected arguments here:
	\c@pgfmanual@pretty@openbracecount=0
	\c@pgfmanual@pretty@curargcount=0
	%
	% and I will accumulate the argument token lists as such here:
	\toksdef\t@pgfmanual@currentarg=0
	\toksdef\t@pgfmanual@allargs=1
	\t@pgfmanual@currentarg={}%
	\t@pgfmanual@allargs={}%
	%
	\pgfmanual@pretty@collectargs@loop
}%

% loops through all tokens, collecting the required number of
% arguments. This involves to track nested braces manually.
% #1 is a single token.
\def\pgfmanual@pretty@collectargs@loop#1{%
	\let\pgfmanual@pretty@collectargs@loop@NEXT=\pgfmanual@pretty@collectargs@loop
	\def\pgfmanual@loc@TMPc{#1}%
	\ifx\pgfmanual@loc@TMPc\pgfmanual@EOI
		\ifnum\c@pgfmanual@pretty@openbracecount>0
			\ifpgfmanualpdfwarnings
				\pgfplots@warning{The pretty printer did not found the closing curly brace!? This will potentially lead to display errors}%
			\fi
		\fi
		\edef\pgfmanual@pretty@collectargs@loop@NEXT{%
			\noexpand\endgroup
			\ifnum\c@pgfmanual@pretty@openbracecount>0
				\noexpand\pgfmanualprettyprinterarghasunmatchedbracestrue
			\fi
			\noexpand\def\noexpand\pgfmanualprettyprinternumcollectedargs{\the\c@pgfmanual@pretty@curargcount}%
			\noexpand\pgfmanualprettyprinterhandlecollectedargs{\the\t@pgfmanual@allargs\the\t@pgfmanual@currentarg}%
			\the\t@afterpgfmanualprettyprinterhandlecollected
			\noexpand\pgfmanual@EOI% <- put this token back into token list!
		}%
	\else
		\ifpgfmanualprettycommentactive
			\t@pgfmanual@currentarg=\expandafter{\the\t@pgfmanual@currentarg#1}%
			\ifx\pgfmanual@loc@TMPc\pgfmanual@pretty@activenl
				\pgfmanualprettycommentactivefalse
			\fi
		\else
			\ifx\pgfmanual@loc@TMPc\pgfmanual@pretty@activepercent
				\t@pgfmanual@currentarg=\expandafter{\the\t@pgfmanual@currentarg#1}%
				\pgfmanualprettycommentactivetrue
			\else
				\ifx\pgfmanual@loc@TMPc\pgfmanual@pretty@lbrace
					\advance\c@pgfmanual@pretty@openbracecount by1
					\ifnum\c@pgfmanual@pretty@openbracecount>1
						\t@pgfmanual@currentarg=\expandafter{\the\t@pgfmanual@currentarg#1}%
					\fi
				\else
					\ifx\pgfmanual@loc@TMPc\pgfmanual@pretty@rbrace
						\advance\c@pgfmanual@pretty@openbracecount by-1
						\ifnum\c@pgfmanual@pretty@openbracecount=0
							% we have one complete argument ready!
							% append it -- with REAL braces if needed:
							\edef\pgfmanual@loc@TMPc{\the\t@pgfmanual@allargs{\the\t@pgfmanual@currentarg}}%
							\t@pgfmanual@allargs=\expandafter{\pgfmanual@loc@TMPc}%
							\t@pgfmanual@currentarg={}%
							%
							% check it we need more arguments:
							\advance\c@pgfmanual@pretty@curargcount by1
							\ifnum\c@pgfmanual@pretty@curargcount=\pgfmanual@loc@csargcount\relax
								\edef\pgfmanual@pretty@collectargs@loop@NEXT{%
									\noexpand\endgroup
									\noexpand\def\noexpand\pgfmanualprettyprinternumcollectedargs{\the\c@pgfmanual@pretty@curargcount}%
									\noexpand\pgfmanualprettyprinterhandlecollectedargs{\the\t@pgfmanual@allargs}%
									\the\t@afterpgfmanualprettyprinterhandlecollected
								}%
							\fi
						\else
							\t@pgfmanual@currentarg=\expandafter{\the\t@pgfmanual@currentarg#1}%
						\fi
					\else
						\ifnum13=\catcode`#1
							% we found a white space (space, TAB or NL) or comment
							\t@pgfmanual@currentarg=\expandafter{\the\t@pgfmanual@currentarg#1}%
						\else
							\ifnum10=\catcode`#1
								% we found a white space (space, TAB
								% or NL) with unexpected catcode
								\t@pgfmanual@currentarg=\expandafter{\the\t@pgfmanual@currentarg#1}%
							\else
								\ifnum\c@pgfmanual@pretty@openbracecount=0
									\ifpgfmanualpdfwarnings
										\begingroup
										\toks4{#1}%
										\pgfplots@warning{The pretty printer did not find the expected \pgfmanual@loc@csargcount\space arguments (only token \the\toks4 )}%
										\endgroup
									\fi
									\edef\pgfmanual@pretty@collectargs@loop@NEXT{%
										\noexpand\endgroup
										\noexpand\def\noexpand\pgfmanualprettyprinternumcollectedargs{\the\c@pgfmanual@pretty@curargcount}%
										\noexpand\pgfmanualprettyprinterhandlecollectedargs{\the\t@pgfmanual@allargs\the\t@pgfmanual@currentarg}%
										\the\t@afterpgfmanualprettyprinterhandlecollected
										\noexpand#1% <- put this token back into token list!
									}%
								\else
									\t@pgfmanual@currentarg=\expandafter{\the\t@pgfmanual@currentarg#1}%
								\fi
							\fi
						\fi
					\fi
				\fi
			\fi
		\fi
	\fi
	\pgfmanual@pretty@collectargs@loop@NEXT
}%

\newif\ifpgfmanualprettyprinterfoundterminator

% Collects tokens inside of a token list up to a single delimiting token, dealing with nested
% catcode-12-braces.
%
% #1: is the end token, the delimiter. It won't be collected.
% #2: is code to invoke after
% \pgfmanualprettyprinterhandlecollectedargs has been invoked.
%
% PRECONDITION:
% 	there is a large token list following
% 	\pgfmanualprettyprintercollectargcount with balanced braces. The braces
% 	have \catcode 12.
% POSTCONDITION:
% 	- the arguments will be collected as
% 		'<result> := <token list>'
% 	and then, \pgfmanualprettyprinterhandlecollectedargs{<result>} will be
% 	invoked. Just define \pgfmanualprettyprinterhandlecollectedargs
% 	properly.
%   Then, #2 will be invoked.
%	- The boolean \ifpgfmanualprettyprinterfoundterminator will be set to true if and only if '#1' has been found.
%
% @see \pgfmanualprettyprintercollectargcount for more details.
%
\def\pgfmanualprettyprintercollectupto#1#2{%
	\pgfmanualprettyprinterarghasunmatchedbracesfalse
	\pgfmanualprettyprinterfoundterminatortrue
	\begingroup
	\def\pgfmanual@loc@delimittoken{#1}%
	\toksdef\t@afterpgfmanualprettyprinterhandlecollected=10
	\t@afterpgfmanualprettyprinterhandlecollected={#2}%
	%
	\let\c@pgfmanual@pretty@openbracecount=\c@pgf@counta
	% I will track open braces here:
	\c@pgfmanual@pretty@openbracecount=0
	%
	% and I will accumulate the argument token lists as such here:
	\toksdef\t@pgfmanual@currentarg=0
	\t@pgfmanual@currentarg={}%
	%
	\pgfmanual@pretty@collectupto@loop
}%

% loops through all tokens, collecting the required number of
% arguments. This involves to track nested braces manually.
% #1 is a single token.
\def\pgfmanual@pretty@collectupto@loop#1{%
	\let\pgfmanual@pretty@collectupto@loop@NEXT=\pgfmanual@pretty@collectupto@loop
	\def\pgfmanual@loc@TMPc{#1}%
	\ifx\pgfmanual@loc@TMPc\pgfmanual@EOI
		\edef\pgfmanual@pretty@collectupto@loop@NEXT{%
			\noexpand\endgroup
			\ifnum\c@pgfmanual@pretty@openbracecount>0
				\noexpand\pgfmanualprettyprinterarghasunmatchedbracestrue
			\fi
			\noexpand\pgfmanualprettyprinterfoundterminatorfalse
			\noexpand\pgfmanualprettyprinterhandlecollectedargs{\the\t@pgfmanual@currentarg}%
			\the\t@afterpgfmanualprettyprinterhandlecollected
			\noexpand\pgfmanual@EOI% <- put this token back into token list!
		}%
	\else
		\ifpgfmanualprettycommentactive
			\t@pgfmanual@currentarg=\expandafter{\the\t@pgfmanual@currentarg#1}%
			\ifx\pgfmanual@loc@TMPc\pgfmanual@pretty@activenl
				\pgfmanualprettycommentactivefalse
			\fi
		\else
			\ifx\pgfmanual@loc@TMPc\pgfmanual@pretty@activepercent
				\t@pgfmanual@currentarg=\expandafter{\the\t@pgfmanual@currentarg#1}%
				\pgfmanualprettycommentactivetrue
			\else
				\ifx\pgfmanual@loc@TMPc\pgfmanual@pretty@lbrace
					\advance\c@pgfmanual@pretty@openbracecount by1
					\t@pgfmanual@currentarg=\expandafter{\the\t@pgfmanual@currentarg#1}%
				\else
					\ifx\pgfmanual@loc@TMPc\pgfmanual@pretty@rbrace
						\advance\c@pgfmanual@pretty@openbracecount by-1
						\t@pgfmanual@currentarg=\expandafter{\the\t@pgfmanual@currentarg#1}%
					\else
						\ifx\pgfmanual@loc@TMPc\pgfmanual@loc@delimittoken
							\ifnum\c@pgfmanual@pretty@openbracecount=0
								% do NOT include the delimit token.
								\edef\pgfmanual@pretty@collectupto@loop@NEXT{%
									\noexpand\endgroup
									\noexpand\pgfmanualprettyprinterhandlecollectedargs{\the\t@pgfmanual@currentarg}%
									\the\t@afterpgfmanualprettyprinterhandlecollected
								}%
							\else
								\t@pgfmanual@currentarg=\expandafter{\the\t@pgfmanual@currentarg#1}%
							\fi
						\else
							\t@pgfmanual@currentarg=\expandafter{\the\t@pgfmanual@currentarg#1}%
						\fi
					\fi
				\fi
			\fi
		\fi
	\fi
	\pgfmanual@pretty@collectupto@loop@NEXT
}%

\def\pgfmanual@pretty@collectargs@finish@cs#1{%
	\def\pgfmanual@pretty@collectargs@finish@cs@hasargs{1}%
	\ifnum\pgfmanualprettyprinternumcollectedargs=0
		\ifpgfmanualprettyprinterarghasunmatchedbraces
		\else
			\def\pgfmanual@pretty@collectargs@finish@cs@hasargs{0}%
		\fi
	\fi
	\if\pgfmanual@pretty@collectargs@finish@cs@hasargs1%
		% report the macro and its arguments:
		\pgfkeysifdefined{/codeexample/prettyprint/cs/\pgfmanual@loc@TMPa/.@cmd}{%
			% oh, we have a separate routine for this macro! Ok, use it:
			\edef\pgfmanual@loc@TMPb{\noexpand\pgfkeysvalueof{/codeexample/prettyprint/cs/\pgfmanual@loc@TMPa/.@cmd}}%
		}{%
			% use the generic routine:
			\edef\pgfmanual@loc@TMPb{\noexpand\pgfkeysvalueof{/codeexample/prettyprint/cs with args/.@cmd}}%
		}%
		% re-insert the backslash:
		\edef\pgfmanual@loc@TMPa{{\pgfmanual@pretty@backslash\pgfmanual@loc@TMPa}}%
		\expandafter\pgfmanual@loc@TMPb\pgfmanual@loc@TMPa #1\pgfeov
	\else
		% Oh. We probably got |\pgfkeys| instead of |\pgfkeys{arg}|
		% re-insert the backslash:
		\edef\pgfmanual@loc@TMPa{{\pgfmanual@pretty@backslash\pgfmanual@loc@TMPa}}%
		\edef\pgfmanual@loc@TMPb{\noexpand\pgfkeysvalueof{/codeexample/prettyprint/cs/.@cmd}}%
		\expandafter\pgfmanual@loc@TMPb\pgfmanual@loc@TMPa\pgfeov
		#1\relax% simply typeset any encountered tokens after the control sequence.
	\fi
}%

% A user macro which pretty prints a set of keys.
%
% If '#1' is NOT an argument for \pgfkeys,
% \pgfmanualprettyprintpgfkeys will try to recognise at least control
% sequences (in the same way as usual). Thus, you can even use this
% method if there *could* be pgfkeys arguments in an automated pretty
% printing environment.
%
% However, '#1' should only have catcode 12 characters with the
% exception of catcode 13 for newlines, spaces and comments.
\long\def\pgfmanualprettyprintpgfkeys#1{%
	\ifpgfmanualprettyenabled
		\begingroup
		\pgfmanualprettyprintstyles
		\pgfmanual@pretty@pgfkeys@loop#1\pgfmanual@EOI
		\endgroup
	\else
		#1%
	\fi
}

\def\pgfmanual@pretty@pgfkeys{%
	\let\pgfmanualprettyprinterhandlecollectedargs=\pgfmanual@pretty@pgfkeys@start
	\pgfmanualprettyprintercollectupto]{\pgfmanual@pretty@mainloop}%%
}%
\long\def\pgfmanual@pretty@pgfkeys@start#1{%
	[%
	\pgfmanual@pretty@pgfkeys@loop#1\pgfmanual@EOI
	]%
}%
% iterates through single tokens, detecting key names and values while
% it goes.
\long\def\pgfmanual@pretty@pgfkeys@loop#1{%
	\def\pgfmanual@loc@TMPa{#1}%
	\ifx\pgfmanual@loc@TMPa\pgfmanual@EOI
		\def\pgfmanual@pretty@pgfkeys@loop@NEXT{\relax}%
	\else
		\def\pgfmanual@pretty@pgfkeys@loop@NEXT{\pgfmanual@pretty@pgfkeys@loop}%
		\ifpgfmanualprettycommentactive
			#1%
		\else
			\ifnum13=\catcode`#1
				% we found a white space (space, TAB or NL) or comment
				#1%
			\else
				\ifx\pgfmanual@loc@TMPa\pgfmanual@pretty@comma%
					%\let\pgfmanual@pretty@pgfkeys@loop@NEXT=\pgfmanual@pretty@pgfkeys@checkforEOI
					,%
				\else
					\ifnum10=\catcode`#1
						% another white space... I thought they'd have
						% catcode 13. doesn't matter.
						#1%
					\else
						\ifx\pgfmanual@loc@TMPa\pgfmanual@pretty@lbrace
							% braces may not occur in the first place -
							% but there are circumstances where it is
							% convenient to deal with them here (when
							% processing arguments of styles)
							% Do it.
							#1%
						\else
							\ifx\pgfmanual@loc@TMPa\pgfmanual@pretty@rbrace
								#1%
							\else
								% found the beginning of a key!
								% We will collect the key name into
								% \toks0.
								\toks0={}%
								%
								% Handle it:
								% FIXME what if we found an opening brace!?
								\def\pgfmanual@pretty@pgfkeys@loop@NEXT{\pgfmanual@pretty@pgfkeys@collectkey #1}%
							\fi
						\fi
					\fi
				\fi
			\fi
		\fi
	\fi
	\pgfmanual@pretty@pgfkeys@loop@NEXT
}%
\def\pgfmanual@pretty@pgfkeys@collectkey#1{%
	\def\pgfmanual@loc@TMPb{#1}%
	\def\pgfmanual@pretty@pgfkeys@collectkey@next{\pgfmanual@pretty@pgfkeys@collectkey}%
	\ifx\pgfmanual@loc@TMPb\pgfmanual@EOI
		% finish key name.
		\edef\pgfmanual@loc@TMPb{\the\toks0 }%
		\ifx\pgfmanual@loc@TMPb\pgfutil@empty
		\else
			\expandafter\pgfmanual@pretty@pgfkeys@processkey\expandafter{\the\toks0 }%
		\fi
		\def\pgfmanual@pretty@pgfkeys@collectkey@next{\relax}%
	\else
		\ifx\pgfmanual@loc@TMPb\pgfmanual@pretty@eq
			% finish key name.
			\expandafter\pgfmanual@pretty@pgfkeys@processkey\expandafter{\the\toks0 }%
			#1%
			% now, we do the same with the value - we collect it into
			% \toks0.
			\toks0={}%
			\def\pgfmanual@pretty@pgfkeys@collectvalue@hasconsumedspaces{0}%
			\def\pgfmanual@pretty@pgfkeys@collectkey@next{\pgfmanual@pretty@pgfkeys@collectvalue}%
		\else
			\ifx\pgfmanual@loc@TMPb\pgfmanual@pretty@comma
				% finish key name.
				\expandafter\pgfmanual@pretty@pgfkeys@processkey\expandafter{\the\toks0 }%
				#1%
				\def\pgfmanual@pretty@pgfkeys@collectkey@next{\pgfmanual@pretty@pgfkeys@loop}%
			\else
				\ifx\pgfmanual@loc@TMPb\pgfmanual@pretty@activenl
					% finish key name before newline. This is not
					% necessarily as in TeX, but its simpler here
					% because we don't need special cases for comments
					% and we don't need to gobble following white
					% spaces.
					\expandafter\pgfmanual@pretty@pgfkeys@processkey\expandafter{\the\toks0 }%
					#1%
					\def\pgfmanual@pretty@pgfkeys@collectkey@next{\pgfmanual@pretty@pgfkeys@loop}%
				\else
					\ifx\pgfmanual@loc@TMPb\pgfmanual@pretty@backslash
						% This is confusing. I simply try to invoke
						% the control sequence code and recover as
						% good as possible. Let's see if that's
						% useful.
						\edef\pgfmanual@loc@TMPb{\the\toks0 }%
						\ifx\pgfmanual@loc@TMPb\pgfutil@empty
						\else
							\expandafter\pgfmanual@pretty@pgfkeys@processkey\expandafter{\the\toks0 }%
						\fi
						%
						% we found the start of a control sequence '\command':
						%
						% collect the control sequence name into
						% \pgfmanual@loc@TMPa, but without the backslash:
						\begingroup
						\let\pgfmanual@loc@TMPa=\pgfutil@empty
						\def\afterpgfmanual@pretty@cs{%
							\endgroup
							\pgfmanual@pretty@pgfkeys@loop
						}%
						\def\pgfmanual@pretty@pgfkeys@collectkey@next{\pgfmanual@pretty@collectcs@loop}%
					\else
						\ifx\pgfmanual@loc@TMPb\pgfmanual@pretty@lbrace
							% Braces in key names are allowed (if they
							% are not the first character of a key
							% name)
							\def\pgfmanualprettyprinterhandlecollectedargs##1{%
								\toks1={##1}%
								\edef\pgfmanual@loc@TMPc{%
									\the\toks0 \pgfmanual@pretty@lbrace \the\toks1
									\ifpgfmanualprettyprinterarghasunmatchedbraces
									\else
										\pgfmanual@pretty@rbrace
									\fi
								}%
								\toks0=\expandafter{\pgfmanual@loc@TMPc}%
							}%
							\def\pgfmanual@pretty@pgfkeys@collectkey@next{%
								\pgfmanualprettyprintercollectargcount{1}{\pgfmanual@pretty@pgfkeys@collectkey}%
								#1%
							}%
						\else
							\ifx\pgfmanual@loc@TMPb\pgfmanual@pretty@rbrace
								\expandafter\pgfmanual@pretty@pgfkeys@processkey\expandafter{\the\toks0 }%
								#1%
								\def\pgfmanual@pretty@pgfkeys@collectkey@next{\pgfmanual@pretty@pgfkeys@loop}%
							\else
								\toks0=\expandafter{\the\toks0 #1}%
							\fi
						\fi
					\fi
				\fi
			\fi
		\fi
	\fi
	\pgfmanual@pretty@pgfkeys@collectkey@next
}%
\def\pgfmanual@pretty@pgfkeys@collectvalue#1{%
	\def\pgfmanual@loc@TMPb{#1}%
	\let\pgfmanual@pretty@pgfkeys@collectvalue@next=\pgfmanual@pretty@pgfkeys@collectvalue
	\ifx\pgfmanual@loc@TMPb\pgfmanual@EOI
		% the key value is empty.
		%\expandafter\pgfmanual@pretty@pgfkeys@processvalue\expandafter{\the\toks0 }%
		\let\pgfmanual@pretty@pgfkeys@collectvalue@next=\relax
	\else
		\def\pgfmanual@pretty@isconsumed{0}%
		\if\pgfmanual@pretty@pgfkeys@collectvalue@hasconsumedspaces0%
			\ifnum13=\catcode`#1
				% we found a white space (space, TAB or NL)
				#1%
				\def\pgfmanual@pretty@isconsumed{1}%
			\else
				\ifnum10=\catcode`#1
					% another white space... I thought they'd have
					% catcode 13. doesn't matter.
					\def\pgfmanual@pretty@isconsumed{1}%
					#1%
				\fi
			\fi
		\fi
		\if\pgfmanual@pretty@isconsumed0%
			\def\pgfmanualprettyprinterhandlecollectedargs##1{\pgfmanual@pretty@pgfkeys@processvalue{##1}}% the braces will be handled separately.
			\def\pgfmanual@pretty@pgfkeys@collectvalue@next{%
				\pgfmanualprettyprintercollectupto,{\afterpgfmanual@pretty@collectargs@finish@value}#1%
			}%
		\fi
	\fi
	\pgfmanual@pretty@pgfkeys@collectvalue@next
}%
\def\afterpgfmanual@pretty@collectargs@finish@value{%
	\pgfutil@ifnextchar\pgfmanual@EOI{%
		\pgfmanual@pretty@pgfkeys@loop
	}{%
		\pgfmanual@pretty@pgfkeys@loop,% re-insert the gobbled comma here!
	}%
}%

{\catcode`\^^M=13 \catcode`\ =13\relax\catcode`\%=13\relax\gdef\pgfmanual@pretty@restorespaces{\def%{\pgfmanual@pretty@percent}\def^^M{\pgfmanual@pretty@space}\def {\pgfmanual@pretty@space}}}

% #1: the key's name
\def\pgfmanual@pretty@pgfkeys@processkey#1{%
	\begingroup
		\pgfmanual@pretty@restorespaces
		% remember this key name! It is used when we are dealing with
		% its value later... (if it has a value)
		\xdef\pgfmanual@pretty@pgfkeys@collectkey@keyname{#1}%
	\endgroup
	%
	% check if it is a handled key. FIXME: this could be done in a
	% rigorous way! This here assumes that all key handlers start with
	% '.' and the '.' occurs never right after a '/' otherwise:
	\def\pgfmanual@loc@TMPa{\pgfutil@in@{/.}}%
	\expandafter\pgfmanual@loc@TMPa\expandafter{\pgfmanual@pretty@pgfkeys@collectkey@keyname}%
	\ifpgfutil@in@
		% split into key name and handler name...
		\def\pgfmanual@pretty@splitit##1/.##2\relax{%
			\def\pgfmanual@loc@TMPb{{##1}{.##2}}%
		}%
		\expandafter\pgfmanual@pretty@splitit\pgfmanual@pretty@pgfkeys@collectkey@keyname\relax
		%
		% report key name AND handler:
		\def\pgfmanual@loc@TMPa{\pgfkeysvalueof{/codeexample/prettyprint/key name with handler/.@cmd}}%
		\expandafter\pgfmanual@loc@TMPa\pgfmanual@loc@TMPb\pgfeov
	\else
		% report key name ...
		\def\pgfmanual@loc@TMPa{\pgfkeysvalueof{/codeexample/prettyprint/key name/.@cmd}}%
		\expandafter\pgfmanual@loc@TMPa\pgfmanual@pretty@pgfkeys@collectkey@keyname\pgfeov
	\fi
}
\def\pgfmanual@pretty@pgfkeys@processvalue#1{%
	% report or process the value, depending on the keyname.
	% After all, it *may* be a style which needs to be pretty printed
	% as well.
	\pgfkeysifdefined{/codeexample/prettyprint/key value/\pgfmanual@pretty@pgfkeys@collectkey@keyname/.@cmd}{%
		% oh, we have a separate routine for this macro! Ok, use it:
		\edef\pgfmanual@loc@TMPb{\noexpand\pgfkeysvalueof{/codeexample/prettyprint/key value/\pgfmanual@pretty@pgfkeys@collectkey@keyname/.@cmd}}%
	}{%
		% use the generic routine:
		\edef\pgfmanual@loc@TMPb{\noexpand\pgfkeysvalueof{/codeexample/prettyprint/key value/.@cmd}}%
	}%
	\edef\pgfmanual@loc@TMPa{{\pgfmanual@pretty@pgfkeys@collectkey@keyname}}%
	\expandafter\pgfmanual@loc@TMPb\pgfmanual@loc@TMPa{#1}\pgfeov
}
